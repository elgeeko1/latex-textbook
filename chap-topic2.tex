\chapter{Topic 2}\label{chap:topic2}

\section{Introduction}\label{sec:topic2}
This topic includes figures and exercises.

%%%%%%%%%%%%%%%%%%%%%%%%%%%%%%%%%
%          EXERCISES            %
%%%%%%%%%%%%%%%%%%%%%%%%%%%%%%%%%
\section{Topic 2 Exercises}\label{sec:topic2-exercises}
A few example exercises. Note that solutions appear only if \textbackslash SOLUTIONS is set to true in book.tex.

\begin{asparaenum}
\item Why do grocery stores sell hot dogs in packs of 10 and hot dog buns in packs of 8?

    \begin{solution}
        It's a conspiracy.
    \end{solution}

\item Let $\vec{a}_g=(a_{g_x},a_{g_y},a_{g_z})$ be the specific force of gravity as measured by an accelerometer on a mobile robot. You may assume the accelerometer has been calibrated and is in units of g.
    \begin{compactenum}
    \item Find $\vec{a}_g$ when the robot is on level ground. Verify $\norm{\vec{a}_g}=1g$.
        \begin{solution}
        \begin{equation*}
        \vec{a}_g = 1g\begin{bmatrix}
            0 \\
            0 \\
            1
            \end{bmatrix}
        \end{equation*}

        \begin{equation*}
        \begin{aligned}
        \norm{\vec{a}_g}&=1g\sqrt{0^2+0^2+1^2} \\
        &=1g
        \end{aligned}
        \end{equation*}
        \end{solution}

    \item The robot is placed on a hill with inclination $\theta_\text{I}\in\left[0,\tfrac{\pi}{2}\right]$ in radians. Find $\vec{a}_g$ when the robot is oriented directly uphill. Verify $\norm{\vec{a}_g}=1g$.
        \begin{solution}
        \begin{equation*}
        \vec{a}_g=1g
            \begin{bmatrix}
            \sin(\theta_\text{I}) \\
            0 \\
            \cos(\theta_\text{I})
            \end{bmatrix}
        \end{equation*}

        \begin{equation*}
        \begin{aligned}
        \norm{\vec{a}_g}&= 1g\sqrt{\sin^2(\theta_\text{I})+0^2+\cos^2(\theta_\text{I})}
        \\ &= 1g
        \end{aligned}
        \end{equation*}
        \end{solution}

    \item The robot is placed on a hill with inclination $\theta_\text{I}\in\left[0,\tfrac{\pi}{2}\right]$ in radians. Find $\vec{a}_g$ when the robot is rotated $\theta$ radians clockwise from uphill orientation. Verify $\norm{\vec{a}_g}=1g$.
        \begin{solution}
        \begin{equation*}
        \vec{a}_g=1g\begin{bmatrix}
            \sin(\theta_\text{I})\cos(\theta) \\
            \sin(\theta_\text{I})\sin(\theta) \\
            \cos(\theta_\text{I})
            \end{bmatrix}
        \end{equation*}

        \begin{equation*}
        \begin{aligned}
        \norm{\vec{a}_g}&=1g\sqrt{\sin^2(\theta_\text{I})\cos^2(\theta)+\sin^2(\theta_\text{I})\sin^2(\theta)+\cos^2(\theta_\text{I})} \\ &= 1g\sqrt{\sin^2(\theta_\text{I})\left[\cos^2(\theta)+\sin^2(\theta)\right]+\cos^2(\theta_\text{I})} \\ &= 1g\sqrt{\sin^2(\theta_\text{I})+\cos^2(\theta_\text{I})} \\ &= 1g
        \end{aligned}
        \end{equation*}
        \end{solution}
    \end{compactenum}
\end{asparaenum}


%%%%%%%%%%%%%%%%%%%%%%%%%%%%%%%%%%
%     WiiMote                    %
%%%%%%%%%%%%%%%%%%%%%%%%%%%%%%%%%%
\section{Introducing the WiiMote}\label{sec:topic2-wiimote}
This section shows various types of references, including references within this document, to an external website, and to an exernal PDF (remote or on disk).

%%%%%%%%%%%%%%%%%%%%%%%%%%%%%%%%%%%%%%%
%          PRELAB READING             %
%%%%%%%%%%%%%%%%%%%%%%%%%%%%%%%%%%%%%%%
\subsection{Prelab Reading}
\begin{compactitem}
\item \secref{sec:topic2-wiimote}
\item \linkref{wiki:Wii_Remote}
\item \linkref{WiiBrew:12:WiiMote}
\item \docref{BluetoothSpecialInterestGroup:03:BluetoothHIDProfile}
    \begin{compactitem}
        \item \docpageref{BluetoothSpecialInterestGroup:03:BluetoothHIDProfile}{18}{\S ``Introduction to the HID Protocol''}, p. 18-23
        \item \docpageref{BluetoothSpecialInterestGroup:03:BluetoothHIDProfile}{57}{\S ``BT HID Transaction Header''}, p. 57-64
    \end{compactitem}
\item \docref{Fisher:10:UsingAnAccelerometerForInclinationSensing}
\item \docref{Ozyagcilar:11:ImplementingCompass}
\end{compactitem}


%%%%%%%%%%%%%%%%%%%%%%%%%%%%%%%%%%%%%%%
%          LAB PROCEDURE              %
%%%%%%%%%%%%%%%%%%%%%%%%%%%%%%%%%%%%%%%
\subsection{Lab Exercises}
This is an example of laboratory exercises using a Nintendo WiiMote.

Required equipment:
\begin{compactitem}
\item Nintendo Wii Remote
\end{compactitem}

\begin{asparaenum}
\item \textbf{Plot the uncalibrated accelerometer signal from the WiiMote:} after pairing, open and run \consoleinline{WiiMote Interface.exe}.

    \begin{compactenum}
    \item What are the values of the x and y axes when the WiiMote is at rest on level ground?
        \begin{solution}
        This will vary from device to device, and the grade on which the WiiMote rests, but 0g should produce output around 128 ADC units.
        \end{solution}

    \item What is the value of the z axis when the WiiMote is at rest on level ground? What quantity is being measured?
        \begin{solution}
        This will vary from device to device. At rest the z axis should produce output around 155 ADC units. This is a measurement of gravity, 1g.
        \end{solution}
    \end{compactenum}

\item \textbf{Measure sensitivity and bias of the accelerometer:} Using the signal plots, determine the sensitivity and bias of the sensor.

    \hint{Will it help to measure one of these quantities first before measuring the second?}

    \begin{compactenum}
    \item What bias did you record?
        \begin{solution}
        The bias of the accelerometer is typically around 128 ADC units.
        \end{solution}

    \item What sensitivity did you record?
        \begin{solution}
        The sensitivity of the WiiMote accelerometer is typically around 25 ADC units per g. A common mistake is to measure sensitivity without compensating for the bias, which produces an (incorrect) sensitivity around 25 + 128 ADC units. The accelerometer will not properly scale to $\pm 1g$ if this error is made.
        \end{solution}
    \end{compactenum}

    \item Provide the relevant MATLAB code and screenshots of changes (if any) to the block diagram.
        \begin{solution}
        The MATLAB code that performs the above calibration equation follows.

\begin{lstlisting}[language=matlab]
% calibrate to units of g-forces (g)
acceleration = (sensor - bias) ./ sensivity;

% magnitude (in g)
mag = norm(acceleration);
\end{lstlisting}
        \end{solution}

\item \textbf{Measure Pitch and Roll:} Measure the pitch and roll of the WiiMote.

    \begin{compactenum}
    \item What equations did you use to calculate pitch and roll?
        \begin{solution}
        The solution follows from \cite{Ozyagcilar:11:ImplementingCompass}, Eqn. 13 and Eqn. 15, where $G_{px}$ and $G_{py}$ are swapped. Given acceleration $\vec{s}=(s_x,s_y,s_z)\in\reals^3$,
        \begin{equation*}
        \begin{aligned}
        \text{roll}\triangleq\phi&=\tan^{-1}\left(\frac{s_x}{s_z}\right) \\
        \text{pitch}\triangleq\theta&=\tan^{-1}\left(\frac{-s_y}{s_x\sin(\phi)+s_z\cos(\phi)}\right)
        \end{aligned}
        \end{equation*}
        \end{solution}
    \end{compactenum}
\end{asparaenum}



\section{Figures}\label{sec:topic2-figures}
Here is the logo for the \pointer{GNU} project.

%fig:gnu
\begin{figure}[h]
\centering
\includegraphics[width=0.25\linewidth]{fig/gnu.png}
\figcaption{GNU logo~\cite[]{Suvasa:11:gnulogo}.\label{fig:gnu}}
\end{figure}


\section{Other Useful Definitions}
For inline code, such as \code{[]->int\{return 42;\}} use \textbackslash code .

For long file paths, such as
    \begin{longpath}
    /usr/bin/which
    \end{longpath}
use the \code{longpath} environment.

For a textual representation of a button, use \textbackslash dialogbutton . \dialogbutton{OK}

If you need to fix something in a future version, use \textbackslash fixme. It will show up if \textbackslash DRAFT is set to true in the preamble, and will be hidden otherwise. \fixme{Remind folks that this feature will change layout when text is hidden or shown.}

For a sequence of steps, such as navigating multiple menu levels, use \textbackslash menulist with items separated by colons:
\menulist{File:Import:Document}

For entering commands in a console, use \textbackslash consoleinline or the console environment.
\begin{console}
cd\\
sudo apt-get update\\
sudo apt-get upgrade\\
\end{console}


For code sections, use \textbackslash lstlisting. Default settings are set in the preable.
\begin{lstlisting}
#include <stdio.h>

int main(int argc, char **argv)
{
    printf("Hello, world!\n");
    return 0;
}\end{lstlisting}