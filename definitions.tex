% Authors: Jeff C. Jensen, Edward A. Lee and Sanjit A. Seshia
% Date: 2016-06-08
% Copyright: (c) 2016, all rights reserved.
% License: Creative Commons Attribution 4.0 International (CC BY 4.0).

% Common definitions for this book.

% for justified text
\newcommand*\justify{%
  \fontdimen2\font=0.4em% interword space
  \fontdimen3\font=0.2em% interword stretch
  \fontdimen4\font=0.1em% interword shrink
  \fontdimen7\font=0.1em% extra space
  \hyphenchar\font=`\-% allowing hyphenation
}

% Blocks of code and variables
\newcommand{\code}[1]{{\fontfamily{\ttdefault}\selectfont\small #1}}

% Math symbols.
% reals
\newcommand{\reals}{\ensuremath{\mathbb{R}} }
\newcommand{\rr}{\mathbb{R}}
% integers
\newcommand{\ints}{\ensuremath{\mathbb{Z}} }
\newcommand{\ii}{\mathbb{Z}}
\newcommand{\zz}{\mathbb{Z}}
% natural numbers
\newcommand{\nats}{\ensuremath{\mathbb{N}} }
\newcommand{\nn}{\mathbb{N}}
% complex
\newcommand{\complex}{\ensuremath{\mathbb{C}} }

% Colors.
\definecolor{fixmeColor}{rgb}{1.0,0.2,0.2}
\definecolor{solutionColor}{rgb}{1.0,0.9,0.9}

%--------------------------------------------
% Temporary annotations in the text.
\ifthenelse{\boolean{DRAFT}}{
\newcommand{\fixme}[1]{\textbf{\textcolor{fixmeColor}{(FIXME: #1)}}}
}{
\newcommand{\fixme}[1]{}
}

%--------------------------------------------------------------------------
% Solutions and nosolutions environment
% Allow the @ character to be part of a name
%--------------------------------------------------------------------------
% Solutions and nosolutions environment
% Allow the @ character to be part of a name
\makeatletter
\ifthenelse{\boolean{SOLUTIONS}}{
    \newenvironment{solution}{%
        \vspace{0.5\baselineskip}
        \begin{tcolorbox}[breakable,enhanced,colback=pink]
        \textbf{Solution:}
    } {
        \end{tcolorbox}
        \vspace{0.5\baselineskip}
    } % end newenvironment
} { % ifthenelse separator
% Alternative when there are no solutions.
    \newenvironment{solution}{%
        % Collect data into a box and then don't use it.
        \begin{lrbox}{\@tempboxa}
        \begin{minipage}[h] {\linewidth}
        ~\hfill
    }{ % newenvironment separator
        \end{minipage}
        \end{lrbox}
    } % end newenvironment
} % end ifthenelse
% restore the @ character to special status
\makeatother

%--------------------------------------------------------------------------
% Figure captions.
\newcommand{\figcaption}[1]{\small\caption{#1}}

%--------------------------------------------------------------------------
% Define a term, print it in the text, and add to the index
% Usage:
%   \definition{term}
%   \Definition{term}   for the term to appear with first letter in uppercase
%   \acronym{index entry}{acronym}
%
% Link to a term and index the reference
% Usage:
%   \definitionlink{term}
%   \Definitionlink{term}   for the link to appear with first letter in uppercase
%   \definitionlink{acronym}
%
%   Add a noindex to definitionlink commands to skip adding the reference to the index
\newcommand{\definition}[1]{\textbf{#1}\label{defn:#1}\index{#1|textbf}}
\newcommand{\Definition}[1]{\textbf{\makefirstuc{#1}}\label{defn:#1}\index{#1|textbf}}
\newcommand{\acronym}[2]{\textbf{#1} (\definition{#2})\index{#1|see{#2}}}
\newcommand{\definitionlink}[1]{\hyperref[defn:#1]{#1}\index{#1}}
\newcommand{\definitionlinknoindex}[1]{\hyperref[defn:#1]{#1}}
\newcommand{\Definitionlink}[1]{\hyperref[defn:#1]{\makefirstuc{#1}}\index{#1}}
\newcommand{\Definitionlinknoindex}[1]{\hyperref[defn:#1]{\makefirstuc{#1}}}
% For terms being indexed. Just do \pointer{index term}.
% Puts them in standard font face and creates an index entry.
%   arg: The term being defined.
% \newcommand{\pointer}[1]{#1\index{#1}}
\newcommand{\pointer}[1]{#1\index{#1}}
\newcommand{\Pointer}[1]{\makefirstuc{#1}\index{#1}}

% Automatic formatting of code listings
\definecolor{bluekeywords}{rgb}{0.13,0.13,1}
\definecolor{greencomments}{rgb}{0,0.5,0}
\definecolor{redstrings}{rgb}{0.9,0,0}
\lstset{language=C,
showspaces=false,
showtabs=false,
breaklines=true,
numbers=none,
showstringspaces=false,
breakatwhitespace=true,
escapeinside={(*@}{@*)},
commentstyle=\color{greencomments},
keywordstyle=\color{bluekeywords},
stringstyle=\color{redstrings},
basicstyle=\ttfamily\footnotesize,
frame=single
}

% norm operator
\newcommand{\norm}[1]{\lVert#1\rVert}

%registered symbol
\newcommand{\reg}{\textregistered~}

%trademark symbol
\newcommand{\tm}{\texttrademark~}

% for sequences instructing how to navigate software menus, e.g. "start -> all programs ->"
% Usage: \menulist{File:New:Project}
\newcommand{\menulist}[1]{\justify\StrSubstitute{#1}{:}{~$\!\!\rightarrow\!\!$~}[\dummy]{``\dummy''}}

% for referring to parts of an application window or dialog
\newcommand{\dialogbutton}[1]{\ovalbox{#1}}

% for hints
\newcommand{\hint}[1]{\textit{\textbf{Hint:} #1}}

% tight centering for mid-paragraph center lines
\newenvironment{tightcenter}{%
  \setlength\topsep{0pt}
  \setlength\parskip{0pt}
  \begin{center}
}{%
  \end{center}
}

% to print a long path in small text, centered on its own line
% use tabular environment to create non-paragraph-breaking line skips
\newenvironment{longpath}{%
    \begin{tightcenter}
    \begin{tabular}{c}
    ~\\
    \small\justify
} {%
    \\
    ~\\
    \end{tabular}
    \end{tightcenter}
} % end newenvironment

% Console command and environment
% use tabular environment to create non-paragraph-breaking line skips
\newcommand{\consoleinline}[1]{{\texttt{\justify~#1~}}}
\newenvironment{console}{%
    \begin{tightcenter}
    \fontfamily{\ttdefault}\selectfont
    \begin{tabular}{l}
    ~\\
%    \justify
} {%
    \\
    ~\\
    \end{tabular}
    \end{tightcenter}
} % end newenvironment
